%
% File acl2019.tex
%
%% Based on the style files for ACL 2018, NAACL 2018/19, which were
%% Based on the style files for ACL-2015, with some improvements
%%  taken from the NAACL-2016 style
%% Based on the style files for ACL-2014, which were, in turn,
%% based on ACL-2013, ACL-2012, ACL-2011, ACL-2010, ACL-IJCNLP-2009,
%% EACL-2009, IJCNLP-2008...
%% Based on the style files for EACL 2006 by 
%%e.agirre@ehu.es or Sergi.Balari@uab.es
%% and that of ACL 08 by Joakim Nivre and Noah Smith

\documentclass[11pt,a4paper]{article}
\usepackage[hyperref]{acl2019}
\usepackage{times}
\usepackage{latexsym}

\usepackage{url}

\aclfinalcopy % Uncomment this line for the final submission
\def\aclpaperid{1234} %  Enter the acl Paper ID here

%\setlength\titlebox{5cm}
% You can expand the titlebox if you need extra space
% to show all the authors. Please do not make the titlebox
% smaller than 5cm (the original size); we will check this
% in the camera-ready version and ask you to change it back.

\newcommand\BibTeX{B\textsc{ib}\TeX}

\title{Understanding Scanned Receipts}

\author{Eric Melz \\
  300 S Reeves Dr. \\
  Beverly Hills, CA 90212 \\
  \texttt{eric@emelz.com}
}

\date{}

\begin{document}
\maketitle
\begin{abstract}
Tasking machines with understanding receipts can have important
applications such as enabling detailed analytics on purchases,
enforcing expense policies, and inferring patterns of purchase
behavior on large collections of receipts.  In this paper, we focus on
the task of Named Entity Linking (NEL) of 
scanned receipt line items.  Specifically, the task entails
associating shorthand text from OCR’d receipts with a knowledge base
(KB) of grocery products.  For example, the scanned item ``STO BABY
SPINACH'' should be linked to the catalog item labeled ``Simple Truth
Organic\texttrademark Baby Spinach''.  Experiments that employ a variety of
Information Retrieval techniques in combination with statistical
phrase detection shows promise for effective understanding of scanned
receipt data.

\end{abstract}

\section{Introduction}

Tasking machines with understanding receipts can have important
applications such as enabling detailed analytics on purchases,
enforcing expense policies, and inferring patterns of purchase
behavior on large collections of receipts.  In this paper, we focus on
the task of Named Entity Linking~\cite{Hachey:2012} of
scanned receipt line items.  Specifically, the task entails
associating shorthand text from OCR’d receipts with a knowledge base
(KB) of grocery products.  For example, the scanned item ``STO BABY
SPINACH'' should be linked to the catalog item labeled ``Simple Truth
Organic\texttrademark Baby Spinach''.  

\section{Related Work}

Blah blah blah.

\section{Data}

Blah blah blah.

\section{Methodology}

Blah blah blah.

\section{Experiments}

Blah blah blah.

\subsection{Baseline}

Blah blah blah

\subsection{Wildcards}

Blah blah blah

\subsection{Mashed Wildcards}

Blah blah blah

\subsection{Phrases}

Blah blah blah

\subsection{Fuzzy Phrases}

Blah blah blah

\section{Results}

blah

\section{Conclusion}

blah

\bibliography{receipt_nlu}
\bibliographystyle{acl_natbib}

\appendix

\section{Appendices}
\label{sec:appendix}

blah 

\section{Supplemental Material}
\label{sec:supplemental}
blah blah blah

\end{document}
